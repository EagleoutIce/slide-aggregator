\documentclass{article}
\usepackage{pgffor}% für mehrere Seiten
\usepackage{csquotes}% für Anführungszeichen :)
\usepackage{xcolor,pagecolor}% für die Hintergrundfarbe der Seite um ursprüngliche Größe zu zeigen

\def\PageInfo{Diese PDF hat eine ursprüngliche Größe von \the\pdfpageheight~Höhe und \the\pdfpagewidth~Breite (graue Fläche, die im \texttt{auto}-Modus automatisch angepasst wird).}
\parindent=0pt
\parskip=\baselineskip

\begin{document}
% Hier setzte ich die Hintergrundfarbe, damit man die unterschiedlichen Größen auch im finalen Foliensatz sehen kann.
\pagecolor{gray!25!white}
% Hier ändere ich die Größe abhängig vom jobname, der durch das makefile ja nur eine Zahl ist :)
\pdfpageheight \dimexpr14cm+0.25cm*\jobname\relax
\pdfpagewidth \dimexpr26.5cm+0.5cm*\jobname\relax

% Ab hier kommt der Demo-Inhalt.
\begin{center}
    \Huge \texttt{\jobname.pdf}: \content
\end{center}
\PageInfo
\foreach \i in {0,...,\jobname}{%
    \clearpage Die \thepage. Seite in \enquote{\jobname.pdf} (die Reihenfolge bestimme ich in \texttt{two-example.tex}. Da die Seitengröße in dem Beispiel davon abhängt zeige ich so, dass wirklich die Maxima gewhält werden)\par
    \PageInfo
}
\end{document}